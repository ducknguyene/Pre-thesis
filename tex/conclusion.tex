\chapter{Conclusion and Future works}
\section{Conclusion}
The purpose of this thesis was to simulate a Distributed Denial of Service (DDoS) attack using UDP flooding and investigate the application of Suricata in Intrusion Prevention System mode to counter the attack. By leveraging the GNS3 network emulator, the project constructed a virtualized network topology involving multiple attacker (bot) machines and one victim server. The bots were orchestrated through automated scripts to generate attack traffic using hping3, while the Suricata IPS was configured to intercept and analyze traffic directed toward the target.
\\
The project successfully demonstrated the feasibility of deploying Suricata in a real-time prevention role using the NFQUEUE mechanism in conjunction with iptables. Through this configuration, Suricata was capable of detecting the UDP flood pattern and taking preventive action by dropping packets based on custom rule sets. The experiments also highlighted the importance of tuning both Suricata’s configuration and the traffic generation parameters to observe measurable outcomes. In scenarios with reduced packet rates, Suricata performed effectively, logging and blocking nearly all malicious traffic. However, during high-rate attacks, its performance was somewhat diminished, revealing the challenges of maintaining line-speed packet inspection and filtering on resource-constrained virtual machines.
\\
Despite these challenges, the thesis has shown that Suricata can act as an effective line of defense against specific forms of DDoS attacks, particularly when the attack pattern is well-defined, and system resources are adequate. It also underscored the critical importance of monitoring tools such as Wireshark and system logs in evaluating the success of such defensive measures. Ultimately, this work confirms the practicality of open-source tools for cybersecurity experimentation and supports the broader applicability of Suricata in both academic and professional contexts.
\section{Future Works}
While this thesis has successfully demonstrated the simulation of UDP flood attacks and the partial mitigation of such attacks using Suricata, there are several potential directions for future exploration and enhancement. One of the primary areas for further development involves integrating Suricata with a Security Information and Event Management (SIEM) platform such as the ELK Stack or Splunk. This would enhance real-time alert analysis, centralized log storage, and facilitate more comprehensive visualization of network anomalies.
\\
Additionally, future research could explore the simulation of more complex and diverse attack vectors, such as TCP SYN floods, HTTP request floods, or DNS amplification attacks. Implementing a wider variety of attack types would test Suricata’s versatility and its capability to respond to a broader spectrum of threats. Another promising direction involves augmenting Suricata’s rule-based detection system with behavioral or machine learning models to enhance anomaly detection and reduce reliance on static signatures.
\\
Performance benchmarking across different hardware setups or deployment models could also provide deeper insights into how Suricata scales under increasing loads. For instance, deploying Suricata on a dedicated edge device or using hardware acceleration through DPDK or PF\_RING could improve throughput and enable effective inline protection in production environments. Further experimentation with inline deployments on routers or firewalls could explore the feasibility of using Suricata not just on endpoints but also within the core network infrastructure.
\\
Finally, enhancing the automation of attack orchestration and monitoring through scripting or network automation tools would streamline experimentation and make large-scale simulations more manageable. Future works may also consider the implementation of dynamic countermeasures, where Suricata not only detects and drops malicious traffic but also triggers network reconfiguration or dynamic firewall updates in response to active threats.
\\
In summary, while the current project lays the groundwork for understanding Suricata’s operation in an IPS context, there remains considerable scope to expand the system’s capabilities, explore alternative configurations, and address its limitations under real-world attack conditions.